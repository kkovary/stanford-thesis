\chapter{Single-cell mass spectrometry identifies optimization of metabolic efficiency through low variance and high correlation of protein expression}
% Short title that appears in the header of pages within the chapter
\chaptermark{Module analysis of variance and correlation}

\begin{center}
Kyle M. Kovary, Zhibo Zhang, Mary N. Teruel
\end{center}

\vspace*{\fill}

%\begin{flushleft}
%The writing in this chapter contributed to the following publication:
%\end{flushleft}
%\newline
%\newline
%\bibentry{Kovary2018}

\newpage

\section{Abstract}
Protein expression variation leads to phenotypic variance between cells which, for example, in cell signaling and differentiation decisions, can lead to differences in cell fate. Targeted assays have shown correlation in expression of proteins that are part of larger assemblies or regulatory modules. However, there has not been a direct measurement of correlation within and between all protein modules in a single cell. Here we measure variation and correlation of over 1000 proteins in individual cells using quantitative proteomics of individual Xenopus laevis eggs and found that proteins involved in the same metabolic module tend to have low levels of expression variance, as well as high correlation. Markedly, we also identified a meta-logic of correlation that reflects the connection between metabolic modules. Molecular modeling showed that low variance and high correlation within metabolic modules results in higher efficiency of metabolic pathways. Additionally, the meta-analysis shows that related metabolic modules such as oxidative phosphorylation and lipid metabolism have high levels of correlation with one another, likely further maximizing overall efficiency. Our study argues for a control principle whereby coordinated variance and correlation within and between metabolic modules helps cells to increase metabolic efficiency.

\section{Highlights}
\begin{itemize}
\item Carried out large-scale analysis of variation and correlation of protein expression in single cells
\item Found that metabolic pathways tend to have low variation and high co-variation of protein expression, which would allow cells to express proteins at their ideal stochiometric levels that would maximize pathway efficiency.
\item •	Strategy of correlated expression of proteins extends beyond just individual modules and functionally related modules have higher than expected correlations.
\end{itemize}

\section{Introduction}
It is well known that variation in protein expression in single cells is important for adding diversity to functions and fate in populations of genetically homogenous cells (Ahrends et al., 2014; Kovary et al., 2018; Spencer et al., 2009; Suderman et al., 2017).  For example, high amounts of variance can decrease the output of a metabolic pathway but can increase the population level control of a binary signaling pathway. 

However since it has become more and more apparent that the functional units in cells are not an individual protein, but rather groups of proteins working together (e.g. signaling pathways, metabolic pathways, and protein complexes), here referred to as modules, it is important to understand how variation in protein expression affects the cell on a modular level. Since there are multiple proteins in modules, it thus becomes important to understand not just the variation in expression of a single protein but how the variation in expression of individual proteins is correlated.

When quantifying correlation between proteins, it is important to take into account the state of the cell since cell states can result in artifactual protein correlation. For example, not taking into account cell size can lead to misleading measures of correlation since larger cells will on average be expressing more proteins than smaller cells. However, after accounting for cell states that are not of interest, the remaining correlation can be used to understand the regulation of protein within modules as well as related modules. This remaining covariation could be indicative of upstream signaling and regulatory process such as shared transcriptional regulation (Stewart-Ornstein et al., 2012), co-translation (Li et al., 2014; Shi et al., 2017), and co-degradation (Mcshane et al., 2016).

Though the collective noise of proteins expressed in modules is key to understanding cellular behavior, the ability to measure module noise is challenging. Part of the challenge is that such measurements require measuring the abundance of many proteins in the same single cell. In a study carried out in yeast, the expression of 743 GFP-tagged proteins versus 44 RFP-tagged proteins was measured. The correlation of the noise, or “noise regulons” were calculated, and the resulting analysis suggested that these noise regulons can be used to identify signaling modules (Stewart-Ornstein et al., 2012). In vertebrate cells, co-variation of tens of proteins have been measured indirectly using antibodies followed by imaging (Gut et al., 2018), flow cytometry (Gaudet et al., 2012), or Cytof mass cytometry (Bendall et al., 2011). Analysis of the single-cell proteome using mass spectrometry analysis is difficult due to the small size of typical cells and the relative low efficiently of peptide-detection, but emerging techniques have shown that it can be useful for tracking cell state changes over time (Budnik et al., 2018).  

Because of the importance of protein modules rather an individual protein as the functional unit in cells, we need a way to be able to measure protein variation and covariation within modules. Mass spectrometry has been a workhorse for proteomics studies, however given the potential for noise in small sample sizes, single cell mass spectrometry has been rarely used. Here we tested whether using Xenopus laevis eggs, a vertebrate single cell model with a several order magnitude larger size, would enable robust simultaneous measurements of variation and co-variation of proteins on a proteome-wide scale and could overcome the signal-to-noise limitations caused by low sample protein levels in individual mammalian cells. We prepared samples of Xenopus laevis eggs during their first cell cycle following previously published protocols (Kovary et al., 2018) and conducted shotgun proteomics analyses. With this data set we have been able to gain insights into the relationship between protein expression variance, coordinated protein expression, and module variance at a proteomic scale in single cells. We identified classes of proteins that include heteromeric complexes and metabolic pathways that are expressed in such a way that can allow modules that rely on stoichiometric expression to function at higher levels of efficiency. Through coordinated expression (correlation) of proteins in a module with low expression variance, cells can express proteins in given complex or pathway at more stable and stoichiometric levels. Seeing multiple examples of the low variation and high correlated modules suggests that this elegant balancing act may be a general control strategy that allows for finer control of metabolic throughput and controlling the number of potentially formed complexes by expressing constituents closer to their stoichiometrically ideal levels. Through network analysis we show that this strategy extends beyond individual modules and can be used between functionally related modules.

\section{Results}
\subsection{Single cell proteomics reveals global protein expression variability and coordinated expression between protein pairs.}

Multicomponent modules such as heteromeric protein complexes and metabolic and signaling pathways have certain tolerances for noise. Two components of the noise of modules are the expression variance of the individual proteins, as well as the correlation of expression between the proteins. These noise components can be combined using the variance sum law (Equation 1) to quantify the total variance of the module. This value has the potential to be misleading given that the same level of total variance can have vastly different values of variation and covariation (Fig 1A, orange lines). For example, a hypothetical module of two proteins could have a total variance level of 0.25 while the proteins in one scenario have variance and correlation values of 0.25 and 0, and another with variance and correlation values of 0.247 and 0.75 (Fig. 1A-B, blue and red respectively). Our previous work has shown that increased levels of correlation, while having a small impact on total variance, can have huge effects on signaling behavior between single cells (Kovary et al., 2018). In order to understand the impact of noise on modules, it is therefore very important to study not only the variance of protein expression but also the correlation. In order to study the levels of variance and correlation of modules at a large scale, the constituent proteins must be measured simultaneously in single cells.

In order to study the relationship between protein expression variance and correlation of cellular pathways and complexes at the proteome level, we used Xenopus laevis eggs as a single cell model for proteomics. We activated Xenopus laevis eggs with calcium ionophore to initiate the cell cycle. We collected 5 eggs at 5 time points (0, 20, 40, 60, and 80 minutes) across the first cell cycle during which the egg remained a single cell following established protocols (Tsai et al., 2014)  (Fig S1A). Individual eggs were mechanically lysed and centrifuged to deplete the yolk the proteins. The proteins were then digested into peptides, labeled using TMT isobaric tags, multiplexed, and analyzed using a shotgun mass spectrometry approach. The relative abundance of more than 1000 proteins in each egg was quantified. Expression of the measured proteins across the time course of the cell cycle showed no dynamic pattern, as demonstrated by a PCA analysis which showed no discernible clustering of the eggs on cell cycle time (Fig. S1B), in agreement with previous studies of Xenopus laevis eggs at this stage (Presler et al., 2017). This result supports the conclusion that the proteins included in our analysis are likely not regulated by cell cycle processes, consistent with previous results which showed that only classic cell cycle protein regulators such as Cyclin A had significant changes in abundance during the first cell cycle (Kovary et al., 2018).  We thus used all 25 eggs for the analysis without distinguishing timepoints in order to increase our statistical observations.

To determine the expression variability for the measured proteins between single cells, we calculated the coefficient of variation (CV) for each protein across all 25 samples and observed a wide range of variation (Fig. 1C), many of which are consistent with our previous study of variation using targeted mass spectrometry (Fig. SXX). An added benefit to measuring these proteins in parallel is that we are able to calculate coordinated expression of protein pairs at a single cell resolution. Using the Pearson correlation coefficient, we were able to determine the coordinated expression of nearly 2 million protein pairs (Fig. 1D). The distribution of correlation coefficients fit a normal distribution centered around 0, with the majority of protein pairs appearing to not show significant co-regulation (Fig. S2). However, there appeared to be a significant number of protein pairs containing high correlation coefficients, and a clustered heat map showed that many highly co-regulated pairs cluster together (Fig. 1D).

This method allowed us to simultaneously measure the variance of individual proteins as well as the correlation between all protein pairs in single cells (Fig. 1E).

\subsection{Modules have varying levels of expression variance and coordinated expression of proteins}

In order to measure the variability and coordinated expression of proteins within modules, we used the KEGG ID and GO Term databases to categorize proteins into 1112 modules. In order for a module to be included it needed at least 4 proteins that were measured in single cells from our dataset. In order to have a single numeric value capture the overall variance of each module, the average CV of all of the proteins within a module was calculated (Average CV of the Module) (Fig 2A). We also wanted to provide a metric for the internal consistency of the variance of the proteins within a given module. For example, the CVs of proteins in the ribosome module (orange bars, Fig. 2C) are consistently low, but the CVs of the proteins in the GTPase module (purple bars, Fig. 2C) are a mixture of high and low.  By calculating the standard deviation of the variances within each module, we can consider modules that have consistent low or high levels of variance (Internal Consistency, Fig. SXX). This was done for all measured modules, and the module variance across the dataset ranged from 8\% to 34\%, with a median around 15\%. Figure 2B highlights 10 modules that had variances from high and low regions of the distribution. 

Fig 2C shows the protein abundance distributions and their respective CV’s for 10 representative modules that span the distribution of module variance (Fig 2A). For each module, the expression distribution and coefficient of variation for each protein could be analyzed individually. Modules such as the ErbB signaling pathway, large ribosomal subunit, and pentose phosphate pathway consistently have proteins with lower variance relative to the other proteins measured in this dataset. In previous work we discussed the observation and implications of low expression variance in modules like the ERK pathway (Mapk1 and Map2k1 are represented here as part of the ErbB module). It is interesting that metabolic pathways that are important at this stage of development had proteins with CVs that are consistently lower than the median. Pyruvate kinase, and with it glycolysis, has been shown to be largely inactive (Dworkin and Dworkin-Rastl, 1989a) while inhibition of the pentose phosphate pathway will quickly induce apoptosis for Xenopus oocytes and eggs (Nutt et al., 2005). This is likely due to the fact that at this stage of development carbohydrates are largely consumed by the pentose phosphate pathway to produce NADPH, an important cofactor and reducing agent for anabolic processes such as nucleic acid production. Phospholipid metabolism is highly active in Xenopus oocytes and fertilized eggs, with the yolk containing a significant amount of fatty acids that can be used for energy, leaving carbohydrates for NADHP metabolism. Many proteins in the fatty acid degradation module showed lower than average CVs (Dworkin and Dworkin-Rastl, 1989b). The low variance of the pentose phosphate and fatty acid degradation modules led us to hypothesize that enzymes belonging to highly active modules may be regulated in such a way as to reduce variance to reduce potential metabolic bottlenecks (add citations).

In addition to module variance, we set out to measure the average correlation of proteins in each module (average module correlation). To do this the Pearson correlation coefficient between all protein pairs within the module was calculated (Fig. 3A-B). We observed a range of module coordinated expression ranging from -0.17 to 0.43, with a median around 0.01 (Fig 3B-C). The distribution of module coordinated expression was biased towards the positive range, with some being very positive. 

We observed that heteromeric protein complexes such as the large ribosomal subunit and nucleosome had consistently high levels of coordinated expression of member proteins. This observation is consistent with reports of stoichiometric translation of complex members (Li et al., 2014) as well as the degradation of un-complexed nascent proteins (Mcshane et al., 2016). Additionally, we observed high levels of coordinated expression in the pentose phosphate pathway module, with exception of the proteins Tktl2, a protein similar to Tkt which showed high levels of coordinated expression, and Dera, a protein that may not always be relevant to the function of this pathway. 

Again, we observed contrasts between the glycolysis module with the pentose phosphate and fatty acid degradation modules. While the glycolysis module showed inconsistent levels of coordinated expression, the other two active metabolic modules showed consistent high levels of coordinated expression. Mathematically, high levels of coordinated expression increases variation in a pathway. However, models of E. coli metabolic pathways have shown that it can increase growth rates (Labhsetwar et al., 2013). 

\subsection{Metabolic pathways express proteins with lower than average variance and higher than average correlation.}

It has been demonstrated our previous work and others that the levels of protein expression variance and correlation can have effect the behaviors or efficacy of signaling pathways (Kovary et al., 2018; Suderman et al., 2017). In this study we had a unique opportunity to look at the variance (Fig. 2A) and correlation (Fig. 3A) of protein expression at the module level. When these two parameters were plotted together, we saw that there was a population of modules that occupied the low variance and high correlation space (Fig. 4A). This was surprising to us since modeling our previous work showed that high variance and high correlation is optimal for controlling fractional responses of binary signaling pathways of a population of cells, while low variance and low correlation is optimal for analog signaling pathways. We had not considered what pathways might utilize a low variance and high correlation modality.

To identify what broader categories of modules were in this population we used CateGOrizer, a method of GO term classification (Hu et al., 2008). This allowed us to place the modules into broader categories to identify overrepresented categories in the low variance / high correlation quadrant. By comparing this population of modules to the remaining population, we saw that this region was greatly enriched for metabolic pathways (Fig. 4B). 

Metabolic pathways have properties that make them distinct from signaling pathways, including the fact that they can consume, produce, and compete for molecules used in other metabolic pathways. In order to study what advantages this strategy of low variance and high correlation may have for metabolic pathways, we constructed a simple branched metabolic model where a substrate is converted into a product via two enzymes with identical reaction rates that compete with another pathway for the intermediate molecule (Fig. 4C). Using this simple ODE model, we were able to vary the expression variance and correlation of the two enzymes and randomly sampled 1000 cells from these populations (Fig. 4D).

To quantify the efficiency of the pathway, we defined efficiency as the concentration of product produced relative to the sum of the concentrations of the two enzymes. We found that both decreasing the variance of the enzymes as well as increasing the correlation of two enzymes increased efficiency, with noticeable synergistic effects (Fig. 4E). This result becomes intuitive after considering that decreasing variance and increasing correlation results in proteins being produced closer to their ideal stoichiometric levels. In the case of this idealized model, the optimal stoichiometric levels are 1:1, but this strategy could optimize expression for pathways with any stoichiometric requirements. By approaching this optimal ratio, the observed pathway was able to consume the intermediate at a faster rate, allowing it to better compete with the alternative branched pathway and produce the observed product.

\subsection{Network analysis of modules shows coordinated expression of lipid, amino acid, and mitochondrial proteins}

In early development of Xenopus embryos, yolk protein and lipids are a key source of nutrition and energy (Jorgensen et al., 2009). Additionally glycolysis, a principal energetic pathway feeding into the citric acid cycle, has been shown to be inactive at this stage of development (Dworkin and Dworkin-Rastl, 1989a). Since module level coordinated expression of metabolic pathways appeared to increase efficiency, we wondered if this optimization could be observed between modules (e.g. lipid metabolism and the citric acid cycle).

To illustrate the idea of between module correlations, Figures 5A and 5B show correlation matrices of the mitochondrial protein complex with superoxide dismutase (SOD) activity and mitotic nuclear division modules. The mitochondrial proteins correlate strongly with the SOD enzymes, and therefore there is a lack of clustering between these two modules (Fig. 5A). This relationship is intuitive given the role SOD enzymes play in reducing reactive oxygen species (ROS) produced by mitochondria through the TCA cycle. If more active mitochondrial proteins are created, there could be a proportional increase in ROS. By coordinating the expression of these proteins with SOD enzymes, cells would be better able at reducing ROS levels.

The negative correlations between the mitochondrial protein complex and mitotic nuclear division modules results in strong clustering of these two modules individually in the correlation matrix heatmap (Fig. 5B). This striking negative relationship is in line with the fact that at the onset of mitosis, mitochondria undergo fission before being reassembled later in the cell cycle (Lu et al., 2006). Mitochondrial fusion has been linked to increased levels of oxidative phosphorylation (Yao et al., 2019), and there is evidence that oxidative phosphorylation activity decreases entering mitosis (Kang et al., 2019). This negative coordinated expression between mitosis and mitochondrial protein complex modules could be due to decreased oxidative phosphorylation activity due to mitochondrial fission when cells enters mitosis.

These two targeted examples give credence to the concept of measuring between module correlations and led us to develop a method to identify links between modules at a larger scale. We conceptualized a bait-prey framework that considered the pairwise relationship between all measured modules that had no overlapping proteins. We conducted a statistical test where new pairwise correlations of proteins between the modules was assumed to be zero and compared this to the measured correlation coefficients (Fig 5C). This assumption was based on the distribution of correlation coefficients between all measured protein pairs, which was centered around zero (Fig. S1). A p-value was calculated between the observed and null distributions of correlation coefficients that was corrected for multiple hypothesis testing using a false discovery rate method. Interactions between modules were considered to be significant if the corrected p-value was less than 0.05. In order to classify the between module interactions as positive or negative, we compared the measured module level correlation to the null hypothesis, where a greater than null value was classified as positive, and a less than null value was classified as negative. This analysis can be graphically represented as a volcano plot for each module individually (example in Fig. 5D) or as a network of all modules together where each module is a node and the positive or negative interactions between each module are edges (Figs. 5E-F).

The interaction network for the oxidative phosphorylation module is highlighted in Figure 5D. Each point represents a test between the oxidative phosphorylation module with all other measured modules. The points highlighted in orange are positive interaction modules, and the points highlighted in blue are negative interaction modules. Interestingly, lipid and amino acid degradation modules were found to have positive interactions with oxidative phosphorylation, whereas carbohydrate and glycolysis modules were found to have negative interactions with oxidative phosphorylation. This is strong evidence that correlated expression of proteins extends beyond individual modules and in fact can span between connected modules. Given that yolk is a major source of energy and resources that is consumed over the course of embryogenesis, the coordinated expression of metabolic pathways that metabolize lipids and amino acids into molecules that are fed into the citric acid cycle with the members of that cycle would be a powerful strategy to increase metabolic efficiency (Fig. 4E). Recent work has shown that pathway specific mRNAs can be co-translated (Shi et al., 2017), and that nuclear-encoded mRNAs can be localized to the mitochondria to be translated (Tsuboi et al., 2020), two potential strategies that may be utilized here given that lipid and amino acid degradation can happen within mitochondria.

The relationships between all of these modules can be visually represented as a network graph (Figs. 5E-F). Each of these networks is comprised of several neighborhoods, some with connections between them and others that are isolated. For instance, the positive relationship network contained one neighborhood (Fig. 5 D, orange) that was comprised primarily of protein translation machinery and mitochondrial modules, suggesting that there is a strong positive relationship between ribosomal protein expression and mitochondria at this stage of development. This neighborhood is closely related to a second (Fig. 5D, green) that contained oxidative phosphorylation modules that were tightly linked with various metabolic pathways including lipid metabolism and amino acid degradation pathways.  

The negative relationship network provides additional strength to the previous insights based on the positive relationship network (Fig. 5E). Here we see negative correlations between oxidative phosphorylation and carbohydrate metabolic processes. As stated earlier, much of carbohydrate metabolism is focused on production of NADPH through the pentose phosphate pathway rather than on the production of ATP through oxidative phosphorylation in the mitochondria. This provides further evidence that measuring correlation of protein expression in single cells can elucidate metabolic strategies of cells.
