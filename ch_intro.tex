\chapter{Introduction}

Every cell is unique. Not only with itself and all others, but also with itself in the past and future. This is due to the inherent randomness of discrete interactions of molecules within the cell. Since cells function, make decisions, and create new molecules through physical interactions of discrete molecules, this is an inescapable fact of life.

The first evidence of this was found back in the 1940s by xx while investigating the production of virus in infected bacterial cells. He found that the variance in the production of viral particles far exceeded the variance in the size of the bacterial cells. This phenomenon was observed in another system where xx was investigating xx. 

At the turn of the millennium, the human genome was sequenced and scientists were beginning to understand that both the function and variability of cellular systems were greater than the sum of their parts. This was the beginning of the field of Systems Biology, where new methods allowed for the study of cellular pathways as entire systems, encompassing much more complexity. This systems perspective of cell biology led scientist to better understand the sources and behavior of the inherent randomness of gene expression and cellular function. This led to the discover of intrinsic and extrinsic noise.

Variation and covariation in protein expression have been shown to be important for adding diversity to functions and fate in populations of genetically homogenous cells (Ahrends et al., 2014; Kovary et al., 2018; Spencer et al., 2009; Suderman et al., 2017). There have been numerous studies of protein expression variation and covariation, though the impacts of these sources of noise are often studied independently of each other. However, both of these parameters act together to determine the total variance of a system. Additionally, the nature of these two sources of variance can have very different impacts on the behavior or function of different systems. For example, high amounts of variance can decrease the output of a metabolic pathway but can increase the population level control of a binary signaling pathway. On the other hand, high levels of correlated expression, while technically increasing the variance of a system, can increase the output of metabolic pathways and allow for finer control of population level control of binary signaling pathways. The sources of protein expression variation are well characterized and their effects on individual proteins are well documented (Taniguchi et al., 2010)(add others). One source is commonly referred to as intrinsic noise, which arises from the fact that there is an inherent randomness for discrete molecules to interact in a mixed solution, such as a transcription factor or polymerase interacting with a sequence of DNA. The second is commonly referred to as extrinsic noise, which arises from larger differences between cells such as the number of ribosomes, cell size, and cell state (e.g. cell cycle phase or differentiation). The quantification this variation created a dilemma for cell biology, where the observed raw variance of protein expression seemed to imply that cells have a lack of ability to control protein expression at a level that would allow for robust function. Since functional groups of proteins (e.g. signaling pathways, metabolic pathways, and protein complexes), here referred to as modules, are often recognized as functional units in cells, this variation dilemma is further compounded through error propagation. 

More recent advances in single cell methods have allowed for the simultaneous measurement of the expression of mRNA and/or protein in single cells in parallel with cell states. For example, cell size, total protein expression, cell cycle phase, pathway activity, and differentiation can be used to reduce cell state extrinsic noise effects. The effects of these advancements have been twofold. First, observed expression noise is dramatically reduced after accounting for cell states and has begun to put the robustness dilemma to rest. Second, since certain cell states can be accounted for, the correlation of protein expression between cells can be used to extract more biologically interesting information. This is due to the fact that extrinsic variation typically results in positive correlation between proteins, since for example larger cells will on average be expressing more proteins than smaller cells. After accounting for cell states that are not of interest, the remaining correlation can be used to understand the regulation of protein within modules as well as related modules. This remaining covariation could be indicative of upstream signaling and regulatory process such as shared transcriptional regulation (Stewart-Ornstein et al., 2012), co-translation (Li et al., 2014; Shi et al., 2017), and co-degradation (Mcshane et al., 2016). 
